\documentclass{acm_proc_article-sp}
\usepackage[utf8]{inputenc}

\def\inputfig#1{\input #1}
\def\inputtex#1{\input #1}
\def\inputal#1{\input #1}
\def\inputcode#1{\input #1}

\inputtex{logos.tex}
\inputtex{refmacros.tex}
\inputtex{other-macros.tex}

\begin{document}
\title{A modern implementation of the LOOP macro}
\numberofauthors{1}
\author{\alignauthor
Robert Strandh\\
\affaddr{University of Bordeaux}\\
\affaddr{351, Cours de la Libération}\\
\affaddr{Talence, France}\\
\email{robert.strandh@u-bordeaux1.fr}}

\toappear{Permission to make digital or hard copies of all or part of
  this work for personal or classroom use is granted without fee
  provided that copies are not made or distributed for profit or
  commercial advantage and that copies bear this notice and the full
  citation on the first page. Copyrights for components of this work
  owned by others than the author(s) must be honored. Abstracting with
  credit is permitted. To copy otherwise, or republish, to post on
  servers or to redistribute to lists, requires prior specific
  permission and/or a fee. Request permissions from
  Permissions@acm.org.

%  ELS '15, April 20 - 21 2015, London, UK
%  Copyright is held by the owner/author(s). Publication rights licensed to ACM.
%  ACM 978-1-4503-2931-6/14/08\$15.00.???
%  http://dx.doi.org/10.1145/2635648.2635656
}

\maketitle

\begin{abstract}
We describe a modern implementation of the \commonlisp{} \texttt{loop}
macro.  This implementation is part of the \sicl{} project.

We use \emph{combinatory parsing} to recognize \texttt{loop} clauses,
and we use \clos{} for code generation. 
\end{abstract}

\category{D.3.4}{Programming Languages}{Processors}
[Code generation, Run-time environments]

\terms{Algorithms, Languages}

\keywords{\clos{}, \commonlisp{}, Iteration}

\inputtex{sec-introduction.tex}
\inputtex{sec-previous.tex}
\inputtex{sec-our-method.tex}
\inputtex{sec-benefits.tex}
\inputtex{sec-conclusions.tex}

\bibliographystyle{abbrv}
\bibliography{loop}
\end{document}
