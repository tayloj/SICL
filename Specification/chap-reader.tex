\chapter{Reader}

The reader can be implemented in an almost entirely portable way.  

The code for the reader is located in the \texttt{Code/Reader}
directory.  There are two subdirectories: \texttt{Code/Reader/Fast}
and \texttt{Code/Reader/Simple}.  

\section{The fast reader}

The directory \texttt{Code/Reader/Fast} contains the code for the
reader that we ultimately plan to make the main reader for \sysname{}.
However, at the moment, the code for the fast reader is very
incomplete and quite difficult to understand, mainly because it is
over-engineered.  

For one thing, we imagined an reader that would not only give
expressions as a result of parsing text, but also abstract syntax
trees.  The main difference between the two is that we imagined the
abstract syntax tree to contain information about source location as
well as an expression.  Most of the time, the same effect can be
obtained by creating a hash table mapping expressions to source
locations, but this method breaks down for symbols, numbers, and
characters, because those objects typically occur in several different
locations.  An abstract syntax tree would distinguish between the
different occurrences and give the source location for each one.
However, dealing with abstract syntax trees is complicated, especially
when it comes to macros.  Common Lisp macros transform expressions to
other expressions, and it is very hard to recreate an abstract syntax
tree from an expression that was created by a macro.  Furthermore, we
are now convinced that the traditional method works sufficiently well,
as long as the hash table contains \emph{all} occurrences of a
particular expression, and in particular all occurrences of each
symbol, number, and character.  Notice that other objects can occur
several times as well, because of the possible use of the reader
macros \#\# and \#=.

The second reason for the reader to be over-engineered is that we
wanted to create a very fast implementation (and we still do).  The
way we imagined doing this was to create bspecial versions of the
reader according to important parameters such as the input base and
the readtable case.  We still think this is a good idea, but it might
be better to do it in different phases.  Phase 1 would be to create a
reader that works for any value of those parameters.  Phase 2 would be
to create a special version for the most common parameter combination
such as input base 10 and readtable case of \texttt{:upcase}.  Other
specialized versions could be developed later.  The main method of
obtaining good performance is to  provide a very fast tokenizer in the
form of a state machine that builds up the token while it is being
read.  

\section{The simple reader}

The directory \texttt{Code/Reader/Simple} contains the code for a
simple, straightforward reader.  It is currently fairly complete.
Only the macros \texttt{\#\#} and \texttt{\#=} (see
\refSec{sec-reader-portability-problems}) and the \texttt{\#S} macro
are missing.  

We have made no attemp to make this reader fast.  Its main virtue is
its simplicity.  

\section{Reading floating-point numbers}

\section{Portability problems}
\label{sec-reader-portability-problems}

The main problem as far as portability is concerned is with the reader
macros \texttt{\#\#} and \texttt{\#=}.  When an object labeled with
\texttt{\#=} is being read (and so has not been constructed yet) and a
reference to it is being encountered (thus creating a circular
structure), some temporary value must be put in until the object has
been constructed.  Then the object must be scanned for that temporary
value.  However scanning arbitrary objects for some substructure
cannot be done portably.

Another (minor) problem with respect to portability has to do with the
\emph{backquote} facility.  The \hs{} explicitly encourages
implementations to use more specific constructs than the standard
\commonlisp{} functions \texttt{append}, \texttt{list}, and \texttt{cons} as a
result of parsing the \emph{backquote} reader macro.  The reason for
is so that the \emph{pretty printer} can print such forms in a more
convenient way (i.e. using the backquote and comma characters).
However, the result is often an implementation-specific encoding that
would have to be translated into standard \commonlisp{} constructs by the
compiler, for instance by writing macros for this encoding.  Such
macros would then necessarily be implementation specific, which is a
problem for bootstrapping \sysname{}.  

